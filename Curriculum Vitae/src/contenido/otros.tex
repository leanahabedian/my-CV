\section{Otros Conocimientos}

\subsection{Programación}

Al estudiar ciencias de la computación poseo un dominio amplio de la programación. Los 
lenguajes que domino ampliamente son Java, C, C++, assembler, Ruby y Python, aunque mi 
carrera provee facilidad de adaptación a nuevos lenguajes. Experiencia en el uso de xpath y 
expresiones regulares. Actualmente estuve aprendiendo como es el funcionamiento de metodologias
agiles como \textbf{Scrum}, \textbf{Extreme Programing}, \textbf{principios de Lean} y \textbf{Kanban}

Fuera del ámbito académico y empresarial, leí por mi cuenta algunos libros que enseñan a programar en lenguaje Ruby, Ruby on Rails, entre otros, como por ejemplo: \textit{Mr. Neighborly's Humble Little Ruby Book}, de \textit{Jeremy McAnally}; \textit{Ruby on Rails 3 Tutorial} de \textit{Michael Hartl}; y \textit{Thinking in C++ y Thinking in Java} de \textit{Bruce Eckel}.

Además, poseo otros conocimientos que facilitan la programación como la del uso de sistema de control de versiones como \textbf{SVN}, \textbf{Mercurial} y \textbf{Git}.

Por otra parte obtengo gran conocimiento de la programación de bajo nivel y un fuerte estudio de la estructura interna de una computadora de tecnología IA­32 (Intel Arquitecture de 32bits).

Por último, destaco que suelo darle importancia a la buena presentación de mis textos y/o publicaciones, por lo que recomiendo fuertemente y uso $\LaTeX$ para realizarlas. Un claro ejemplo es este Currículum Vitae que usted esta leyendo.


\subsection{Sistemas}

Muchas de las materias de mi carrera tienen como objeto el funcionamiento de los sistemas 
operativos, estudiando por separado cada componente, como el Memory Management Unit (MMU), I/O unit, scheduler, etc. y su funcionamiento en sistemas distribuidos utilizando algoritmos conocidos.

Además, poseo conocimientos teóricos de los protocolos que se usan en las comunicaciones, arquitectura OSI. Protocolos de nivel 2 como CSMA/CD y CSMA/CA. Protocolos de nivel 3 IP, 4 como TCP/IP o UDP y a nivel de aplicación HTTP, SMTP, POP3, entre otros.

Otra área la cual me interesa mucho y relacionada con redes, es el area de seguridad informática, si bien no tengo
experiencia laboral en esto, fueron conceptos que siempre me llamaron la atención. Lei algunos libros y páginas
web donde conocí las normativas SOX, ISO 27001 y Cobit, entre otras.

Finalizando, me gustaría agregar que también soy usuario domestico del sistema operativo Linux distribución Ubuntu desde el 2009.
