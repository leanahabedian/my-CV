\section{Experiencia Laboral}

\begin{tabular}{c|p{12.5cm}}
& \large Investigación en ingeniería del Software, síntesis de controladores \\
\large\textit{mar 2014 - jun 2014} &  \large con enfoque en sistemas auto adaptables. Desarrollé una nueva
técnica en la herramienta MTSA para poder obtener controladores de sistemas \\
\large\textbf{National Institute} & \large cuyo entorno cambia satisfaciendo nuevos requerimientos y respetando el nuevo
entorno sin la necesidad de apagar y prender la máquina. \\
\large\textbf{of Informatics} & \large  Dicha pasantía fue realizada en Tokyo Japón supervisado por
Dr. Nicolas D'Ippolito y Dr. Kenji Tei.\\
\large\textbf{Japón}  & \large La herramienta MTSA esta desarrollada en \textbf{Java} y puede
descargarse en \url{http://sourceforge.net/projects/mtsa}\\
\end{tabular}


\begin{center}
\line(1,0){450}
\end{center}
\begin{tabular}{c|p{12.5cm}}
& \large Dirección General de Gestión Informática. Desarrollo código \textbf{Python}\\
\large\textit{nov 2012 - actualidad} &  \large  (backend) creando sistemas que procesan muchos datos que maneja la nación argentina.\\
\large\textbf{Ministerio} & \large  Por el momento poseo dos proyectos. La asignación de tiempos que cada medio de comunicación otorgará a la nación para espacio\\
\large\textbf{del Interior} & \large  publicitario de campañas electoral de manera equitativa para cada partido político.\\
& \large Además administro y actualizo la base de datos de las transacciones que se realizaron con la tarjeta sube. Estos son millones de registros que deben ser tratados de forma particular para que puedan mostrarse por una página web\\
& \large Utilizo motores \textbf{MySQL} y \textbf{PostgreSQL} para mi trabajo además de manejo de servidores \textbf{linux}. \\
\end{tabular}

\begin{center}
\line(1,0){450}
\end{center}
\begin{tabular}{c|p{12.5cm}}
& \large Empresa que desarrolla software para análisis de vulnerabilidades.\\
\large\textit{mayo 2012 - oct 2012} &  \large Mi trabajo se basó en desarrollar código \textbf{Python} creando un sistema desde el inicio que permitía agregar, remover o editar tareas del archivo \\
\large\textbf{Core Security} & \large  de crontab de UNIX.\\
\large\textbf{Technologies} & \large  Otra tarea que realicé fue la de levantar un servidor web en una máquina virtual con SO \textbf{OpenBSD}. Para esto instalé en esta todos\\
& \large los paquetes necesarios para levantar un servicio web (\textbf{Apache2}, \textbf{PostgreSQL}) \\
\end{tabular}

\begin{center}
\line(1,0){450}
\end{center}
\noindent
\begin{tabular}{c|p{12.5cm}}
& \large Empresa que se dedicaba a realizar vídeos juegos para redes sociales.\\
\large\textit{mayo 2010 - dic 2010} &  \large Me desenvolví como uno de los desarrolladores de un juego particular llamado Fashion World. El patron que se seguía para implementar dicho juego es \textbf{Modelo vista controlador (MVC)} y el lenguaje de programación \textbf{Action Script 2}.\\
\large\textbf{Metrogames S.A.} & \large Mas luego utilicé \textbf{Python} para realizar programas de uso interno de la compañia, que si bien eran pequeños, yo tenia la libertad de como implementarlo y agregarle operaciones utiles para mis compañeros. Algunos de los programas eran para subir codigo de Action Script 2 implementado en windows a un servidor linux o interactuar con paginas web hechas en PHP.
\end{tabular}

\begin{center}
\line(1,0){450}
\end{center}
\noindent
\begin{tabular}{c|p{12.5cm}}
& \large Realicé pasantías para dicha empresa donde cumplí con dos proyectos:\\
\large\textit{feb 2010 - mayo 2010} & \large Obtención de datos de importancia para la empresa, en sitios webs norteamericanos y polacos, utilizando \textbf{XPATH} y \textbf{expresiones regulares}.  Para este proyecto fue necesario interpretar las necesidades y requerimientos del cliente de habla inglesa y polaca, dandonos a entender mediante el idioma inglés.\\
\large\textbf{Intelligenx Inc.:} & \large Además me desenvolví en programación web, front-end con \textbf{PHP}, \textbf{HTML}, base de datos \textbf{MySql}.
\end{tabular}

