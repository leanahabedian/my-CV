%----------------------------------------------------------------------------------------
%	PACKAGES AND OTHER DOCUMENT CONFIGURATIONS
%----------------------------------------------------------------------------------------

\documentclass[11pt,a4paper,roman]{moderncv} % Font sizes: 10, 11, or 12; paper sizes: a4paper, letterpaper, a5paper, legalpaper, executivepaper or landscape; font families: sans or roman

%% for enies y tildes
\usepackage[utf8]{inputenc}
\usepackage[spanish]{babel}

\moderncvstyle % CV theme
\moderncvcolor % load the moderncvcolor.sty

%\usepackage{lipsum} % Used for inserting dummy 'Lorem ipsum' text into the template

\usepackage[scale=0.75]{geometry} % Reduce document margins
%\setlength{\hintscolumnwidth}{3cm} % Uncomment to change the width of the dates column
%\setlength{\makecvtitlenamewidth}{10cm} % For the 'classic' style, uncomment to adjust the width of the space allocated to your name

%----------------------------------------------------------------------------------------
%	NAME AND CONTACT INFORMATION SECTION
%----------------------------------------------------------------------------------------

\firstname{Leandro Ezequiel} % Your first name
\familyname{Nahabedian} % Your last name

% All information in this block is optional, comment out any lines you don't need
\title{Curriculum Vitae}
\address{Pumacahua 422}{Ciudad Autonoma de Buenos Aires, Argentina}
\mobile{+54 11 3779 5599}
%\phone{(000) 111 1112}
%\fax{(000) 111 1113}
\email{leanahabedian@gmail.com}
\homepage{nahabedian.herokuapp.com}{nahabedian.herokuapp.com} % The first argument is the url for the clickable link, the second argument is the url displayed in the template - this allows special characters to be displayed such as the tilde in this example
%\extrainfo{additional information}
\photo[70pt][0.2pt]{img/Lea.jpg} % The first bracket is the picture height, the second is the thickness of the frame around the picture (0pt for no frame)
%\quote{"A witty and playful quotation" - John Smith}

%----------------------------------------------------------------------------------------

\begin{document}

\makecvtitle % Print the CV title

%----------------------------------------------------------------------------------------
%	EDUCATION SECTION
%----------------------------------------------------------------------------------------

\section{Educación}

\cventry{2015--}{Estudiante de Doctorado en Ciencias de la Computación}{}{}{}{}
\cventry{2007--2014}{Licenciado en Ciencias de la Computación}{Universidad de Buenos Aires}{Argentina}{}{}%{\textit{GPA -- 8.0}}{First Class Honours}  % Arguments not required can be left empty
\cventry{2007--2013}{Analista Universitario en Computación}{Universidad de Buenos Aires}{Argentina}{}{}%{\textit{GPA -- 7.5}}{Specialized in }

\section{Tesis de Licenciatura}

\cvitem{Título}{\emph{Hot-Swap: Una técnica para la generación y actualización automática de controladores discretos en
tiempo de ejecución}}
\cvitem{Director}{Dr. Nicol\'as D'Ippolito}
\cvitem{Descripción}{Producir software que opera continuamente es un atributo de calidad muy común y posee muchas
aplicaciones. Por lo tanto, existe una necesidad de idear técnicas que pueden cambiar un sistema sin frenar o
interrumpir su ejecución cuando se produce un cambio en el ambiente y/o en los requerimientos.
\newline{}
En esta tesis, trabajamos el problema de realizar una actualización de controlador dinámicamente cuando la
especificación de un sistema cambia (tanto las asunciones del ambiente como los requerimientos).
\newline{}
Presentamos una solución general que, no solo produce un controlador que satisface la nueva especificación y maneja la
transición de uno al otro, sino que también, a diferencia de los trabajos ya existentes, forzamos al sistema actual a un
estado, el cual, dicha transición puede ocurrir. A su vez, usando síntesis de controladores mostramos como construir
automáticamente un controlador que garantiza que la actualización sucederá, y además, que dicha
actualización será de manera segura.
}

%----------------------------------------------------------------------------------------
%Jamee	WORK EXPERIENCE SECTION
%----------------------------------------------------------------------------------------

\section{Experiencia Académica}

\cventry{Mar 2014 Ene 2014}{Investigador asistente}{National Institute of Informatics}{Tokyo}{Japón}{
Investigación en ingeniería del software en síntesis de controladores con un enfoque en software auto-adaptables.
Desarrollé una nueva técnica en la herramienta MTSA para construir controladores que permiten cambios en el ambiente y
asegura nuevos objetivos sin detener o interrumpir el sistema.
\newline{}%\newline{}
El trabajo realizado en mi tesis de Licenciatura se inicia en esta pasantía que fue llevada a cabo en Tokyo, Japón y
supervisado por el Dr. Nicolás D'Ippolito y el Dr. Kenji Tei. La herramienta MTSA fue desarrollada en \textsc{java} y
usted puede bajarlo del siguiente link \url{http://sourceforge.net/projects/mtsa}.
}

%===============================================

\section{Experiencia Docente}

\cventry{Mar 2009 Nov 2009}{Clases de apoyo}{Freelance}{Buenos Aires}{Argentina}{
Dictado de clases particulares a estudiantes de colegio secundario. Dichas clases eran de matemática, física e inglés.
Estos estudiantes concurrian a colegios especiales para completar sus estudios secundarios.
}

%===============================================

\section{Experiencia Profesional}

\cventry{Nov 2012 Mar 2015}{Project Leader / Developer}{National Ministry of Interior}{Argentina}{}{
Me desenvolví en diferentes tareas que requerían programar en varios lenguajes de programación. Principalmente,
trabajaba con \textsc{python} pero también trabajé con \textsc{java} y \textsc{php}. Mi actividad era básicamente el
desarrollo en back-end.
\newline{}
Luego, trabajé también como Líder de Proyecto. El proyecto que llevé a cabo consistía en asignar el tiempo que los
medios de comunicación ceden a las agrupaciones políticas para campañas electorales del 2013 y el 2015. Nuestro sistema
contaba con una aplicación web que permitía a los medios de comunicación y a las agrupaciones políticas comunicarse
entre ellos, para poder decidir cuales son los spots publicitarios que se utilizaran durante la campaña en cada medio.
}

%------------------------------------------------
\cventry{May 2012 Oct 2012}{Python developer}{Core Security Technologies}{Buenos Aires}{Argentina}{
Compañía que desarrolla software para análisis de vulnerabilidades. Mi trabajo consistió en desarrollar código
\textsc{python} para construir un sistema que permite agregar, remover o editar tareas en el archivo crontab de
\textsc{unix}.
\newline{}
Otro proyecto en el cual estuve involucrado fue inicializar un servidor en una máquina virtual con \textsc{openBSD} para
ser usado como servidor web. Para lograr dicho objetivo tuve que instalar todos los paquetes necesario para construir un
servicio web.
}

%------------------------------------------------
\cventry{May 2010 Dic 2010}{ActionScript2/Python developer}{MetroGames}{Buenos Aires}{Argentina}{
Compañía que desarrolla video juegos para redes sociales. Fui uno de los desarrolladores de un juego particular llamado
Fashion World. El patrón de diseño que utilizábamos para implementar ese juego fue \textsc{mvc} y el lenguaje utilizado
fue\textsc{ActionScript2}.
\newline{}
Además, también trabajé con \textsc{python} para realizar sistemas para uso dentro de la compañía. Estos programas
formaban parte de la infraestructura interna de la compañía. Un ejemplo de estos, es el sistema que se encargaba de
hacer el ``deployment'' de un proyecto.
}

%------------------------------------------------
\cventry{Feb 2010 May 2010}{Web developer}{Intelligenx}{Buenos Aires}{Argentina}{
Realicé una pasantía para esta compañía donde tuve dos proyectos.
\newline{}
El primero fue configurar una aplicación web usando \textsc{xpath} y \textsc{regular expressions}. Realizando dicha
configuración, la herramienta almacenaba en una base de datos información de locales de diferentes compañías situadas en
EEUU y Polonia.
\newline{}
Luego de esto, producimos una página web que usaba estos datos obtenidos para proveer un servicio de páginas amarillas.
Usamos \textsc{java} para crear la aplicación. 
}


%----------------------------------------------------------------------------------------
%	AWARDS SECTION
%----------------------------------------------------------------------------------------

\section{Honors \& Awards}

\cventry{Ago 2012 Ago 2014}{Beca TIC}{Ministerio de Ciencia, Tecnología e Innovación Productiva}{Argentina}{}{
Obtuve una beca por dos años, otorgada por el Ministerio de Ciencia, Tecnología e Innovación Productiva de la República
Argentina para obtener el título de Licenciado en Ciencias de la Computación. Los beneficiarios de este plan fueron
elegidos en base al número de exámenes rendidos, los años emprendidos en el estudio y el promedio de notas obtenidas.
}

%----------------------------------------------------------------------------------------
%	LANGUAGES SECTION
%----------------------------------------------------------------------------------------

\section{Idiomas}

\cvitemwithcomment{Español}{Nativo}{}
\cvitemwithcomment{Inglés}{Bilingue}{First Certificate in English en 2006}

%----------------------------------------------------------------------------------------
%	INTERESTS SECTION
%----------------------------------------------------------------------------------------

%\section{Interests}

%\renewcommand{\listitemsymbol}{-~} % Changes the symbol used for lists

%\cvlistdoubleitem{Guitar}{Music}
%\cvlistdoubleitem{Cooking}{Dancing}
%\cvlistitem{Running}

\end{document}
