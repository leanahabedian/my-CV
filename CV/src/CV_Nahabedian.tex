%----------------------------------------------------------------------------------------
%	PACKAGES AND OTHER DOCUMENT CONFIGURATIONS
%----------------------------------------------------------------------------------------

\documentclass[11pt,a4paper,roman]{moderncv} % Font sizes: 10, 11, or 12; paper sizes: a4paper, letterpaper, a5paper, legalpaper, executivepaper or landscape; font families: sans or roman

%% for enies y tildes
%\usepackage[utf8]{inputenc}
%\usepackage[spanish]{babel}

\moderncvstyle % CV theme
\moderncvcolor % load the moderncvcolor.sty

%\usepackage{lipsum} % Used for inserting dummy 'Lorem ipsum' text into the template

\usepackage[scale=0.75]{geometry} % Reduce document margins
%\setlength{\hintscolumnwidth}{3cm} % Uncomment to change the width of the dates column
%\setlength{\makecvtitlenamewidth}{10cm} % For the 'classic' style, uncomment to adjust the width of the space allocated to your name

%----------------------------------------------------------------------------------------
%	NAME AND CONTACT INFORMATION SECTION
%----------------------------------------------------------------------------------------

\firstname{Leandro Ezequiel} % Your first name
\familyname{Nahabedian} % Your last name

% All information in this block is optional, comment out any lines you don't need
\title{Curriculum Vitae}
\address{Pumacahua 422}{Ciudad Autonoma de Buenos Aires, Argentina}
\mobile{+54 11 3779 5599}
%\phone{(000) 111 1112}
%\fax{}
\email{leanahabedian@gmail.com}
\homepage{nahabedian.herokuapp.com}{nahabedian.herokuapp.com} % The first argument is the url for the clickable link, the second argument is the url displayed in the template - this allows special characters to be displayed such as the tilde in this example
\extrainfo{07--10--1988}
%\extrainfo{additional information}
\photo[70pt][0.2pt]{img/Lea.jpg} % The first bracket is the picture height, the second is the thickness of the frame around the picture (0pt for no frame)
%\quote{"A witty and playful quotation" - John Smith}

%----------------------------------------------------------------------------------------

\begin{document}

\makecvtitle % Print the CV title

%----------------------------------------------------------------------------------------
%	EDUCATION SECTION
%----------------------------------------------------------------------------------------

\section{Educaci\'on}

\cventry{2015--}{Estudiante de Doctorado en Ciencias de la Computaci\'on}{No inscripto aun}{}{}{}
\cventry{2007--2014}{Licenciado en Ciencias de la Computaci\'on}{Universidad de Buenos Aires}{Argentina}{}{}%{\textit{GPA -- 8.0}}{First Class Honours}  % Arguments not required can be left empty
\cventry{2007--2013}{Analista Universitario en Computaci\'on}{Universidad de Buenos Aires}{Argentina}{}{}%{\textit{GPA -- 7.5}}{Specialized in }

\section{Tesis de Licenciatura}

\cvitem{T\'itulo}{\emph{Hot-Swap: Una t\'ecnica para la generaci\'on y actualizaci\'on autom\'atica de controladores discretos en
tiempo de ejecuci\'on}}
\cvitem{Director}{Dr. Nicol\'as D'Ippolito}
\cvitem{Descripci\'on}{Producir software que opera continuamente es un atributo de calidad muy com\'un y posee muchas
aplicaciones. Por lo tanto, existe una necesidad de idear t\'ecnicas que pueden cambiar un sistema sin frenar o
interrumpir su ejecuci\'on cuando se produce un cambio en el ambiente y/o en los requerimientos.
\newline{}
En esta tesis, trabajamos el problema de realizar una actualizaci\'on de controlador din\'amicamente cuando la
especificaci\'on de un sistema cambia (tanto las asunciones del ambiente como los requerimientos).
\newline{}
Presentamos una soluci\'on general que, no solo produce un controlador que satisface la nueva especificaci\'on y maneja la
transici\'on de uno al otro, sino que tambi\'en, a diferencia de los trabajos ya existentes, forzamos al sistema actual a un
estado, el cual, dicha transici\'on puede ocurrir. A su vez, usando s\'intesis de controladores mostramos como construir
autom\'aticamente un controlador que garantiza que la actualizaci\'on suceder\'a, y adem\'as, que dicha
actualizaci\'on ser\'a de manera segura.
}

%----------------------------------------------------------------------------------------
%	WORK EXPERIENCE SECTION
%----------------------------------------------------------------------------------------

\section{Experiencia Acad\'emica}

\subsection{Participaci\'on en Proyectos}

\cventry{Mar 2014 Jun 2014}{Investigador asistente}{National Institute of Informatics}{Tokyo}{Jap\'on}{
Investigaci\'on en ingenier\'ia del software en s\'intesis de controladores con un enfoque en software auto-adaptables.
Desarroll\'e una nueva t\'ecnica en la herramienta MTSA para construir controladores que permiten cambios en el ambiente y
asegura nuevos objetivos sin detener o interrumpir el sistema.
\newline{}%\newline{}
El trabajo realizado en mi tesis de Licenciatura se inicia en esta pasant\'ia que fue llevada a cabo en Tokyo, Jap\'on y
supervisado por el Dr. Nicol\'as D'Ippolito y el Dr. Kenji Tei. La herramienta MTSA fue desarrollada en \textsc{java} y
usted puede bajarlo del siguiente link \url{http://sourceforge.net/projects/mtsa}.
}

\subsection{Formaci\'on de Recursos Humanos}

\cventry{Ene 2015 actualidad}{Director de Tesis de Licenciatura}{Universidad de Buenos Aires}{Argentina}{}{
Dirijo a Ivan Pasquini, quien se est\'a involucrando en resolver problemas de s\'intesis de controladores con
observaci\'on parcial, para llevar a cabo esta tesis. Tenemos esperado finalizar dicha tesis para agosto de 2015 debido
a que ya estamos ultimando los \'ultimos detalles.
\newline{}
El trabajo presentar\'a una soluci\'on novedosa para problemas donde se desconoce la totalidad de las propiedades del
ambiente. La estrategia es explorar dicho ambiente de una manera inteligente para obtener solamente la informaci\'on
necesaria para resolver el problema. Dicho trabajo tiene aplicaciones en la robótica, en las redes de comunicaci\'on y
otras.
}

%===============================================

\section{Experiencia Docente}

\cventry{Mar 2009 Nov 2009}{Clases de apoyo}{Freelance}{Buenos Aires}{Argentina}{
Dictado de clases particulares a estudiantes de colegio secundario. Dichas clases eran de matem\'atica, f\'isica e ingl\'es.
Estos estudiantes concurrian a colegios especiales para completar sus estudios secundarios.
}

%===============================================

\section{Experiencia Profesional}

\cventry{Nov 2012 actualidad}{L\'ider de proyecto / Desarrollador}{Ministerio del Interior y Transporte}{Argentina}{}{
Me desenvolv\'i en diferentes tareas que requer\'ian programar en varios lenguajes de programaci\'on. Principalmente,
trabajaba con \textsc{python} pero tambi\'en trabaj\'e con \textsc{java} y \textsc{php}. Mi actividad era b\'asicamente el
desarrollo en back-end.
\newline{}
Luego, trabaj\'e tambi\'en como L\'ider de Proyecto. El proyecto que llev\'e a cabo consist\'ia en asignar el tiempo que los
medios de comunicaci\'on ceden a las agrupaciones pol\'iticas para campa\~nas electorales del 2013 y el 2015. Nuestro sistema
contaba con una aplicaci\'on web que permit\'ia a los medios de comunicaci\'on y a las agrupaciones pol\'iticas comunicarse
entre ellos, para poder decidir cuales son los spots publicitarios que se utilizar\'an durante la campa\~na en cada medio.
}

%------------------------------------------------
\cventry{May 2012 Oct 2012}{Desarrollador Python}{Core Security Technologies}{Buenos Aires}{Argentina}{
Compa\~n\'ia que desarrolla software para an\'alisis de vulnerabilidades. Mi trabajo consisti\'o en desarrollar c\'odigo
\textsc{python} para construir un sistema que permite agregar, remover o editar tareas en el archivo crontab de
\textsc{unix}.
\newline{}
Otro proyecto en el cual estuve involucrado fue inicializar un servidor en una m\'aquina virtual con \textsc{openBSD} para
ser usado como servidor web. Para lograr dicho objetivo tuve que instalar todos los paquetes necesario para construir un
servicio web.
}

%------------------------------------------------
\cventry{May 2010 Dic 2011}{Desarrollador ActionScript2/Python}{MetroGames}{Buenos Aires}{Argentina}{
Compa\~n\'ia que desarrolla video juegos para redes sociales. Fui uno de los desarrolladores de un juego particular llamado
Fashion World. El patr\'on de dise\~no que utiliz\'abamos para implementar ese juego fue \textsc{mvc} y el lenguaje utilizado
fue\textsc{ActionScript2}.
\newline{}
Adem\'as, tambi\'en trabaj\'e con \textsc{python} para realizar sistemas para uso dentro de la compa\~n\'ia. Estos programas
formaban parte de la infraestructura interna de la compa\~n\'ia. Un ejemplo de estos, es el sistema que se encargaba de
hacer el ``deployment'' de un proyecto.
}

%------------------------------------------------
\cventry{Feb 2010 May 2010}{Desarrollador Web}{Intelligenx}{Buenos Aires}{Argentina}{
Realic\'e una pasant\'ia para esta compa\~n\'ia donde tuve dos proyectos.
\newline{}
El primero fue configurar una aplicaci\'on web usando \textsc{xpath} y \textsc{regular expressions}. Realizando dicha
configuraci\'on, la herramienta almacenaba en una base de datos informaci\'on de locales de diferentes compa\~n\'ias situadas en
EEUU y Polonia.
\newline{}
Luego de esto, producimos una p\'agina web que usaba estos datos obtenidos para proveer un servicio de p\'aginas amarillas.
Usamos \textsc{java} para crear la aplicaci\'on. 
}


%----------------------------------------------------------------------------------------
%	AWARDS SECTION
%----------------------------------------------------------------------------------------

\section{Premios y Becas}

\cventry{Ago 2012 Ago 2014}{Beca TIC}{Ministerio de Ciencia, Tecnolog\'ia e Innovaci\'on Productiva}{Argentina}{}{
Obtuve una beca por dos a\~nos, otorgada por el Ministerio de Ciencia, Tecnolog\'ia e Innovaci\'on Productiva de la Rep\'ublica
Argentina para obtener el t\'itulo de Licenciado en Ciencias de la Computaci\'on. Los beneficiarios de este plan fueron
elegidos en base al n\'umero de ex\'amenes rendidos, los a\~nos emprendidos en el estudio y el promedio de notas obtenidas.
}

%----------------------------------------------------------------------------------------
%	LANGUAGES SECTION
%----------------------------------------------------------------------------------------

\section{Idiomas}

\cvitemwithcomment{Espa\~nol}{Nativo}{}
\cvitemwithcomment{Ingl\'es}{Bilingue}{First Certificate in English en 2006}

%----------------------------------------------------------------------------------------
%	INTERESTS SECTION
%----------------------------------------------------------------------------------------

%\section{Interests}

%\renewcommand{\listitemsymbol}{-~} % Changes the symbol used for lists

%\cvlistdoubleitem{Guitar}{Music}
%\cvlistdoubleitem{Cooking}{Dancing}
%\cvlistitem{Running}

\end{document}
