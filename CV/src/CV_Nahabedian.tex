%----------------------------------------------------------------------------------------
%	PACKAGES AND OTHER DOCUMENT CONFIGURATIONS
%----------------------------------------------------------------------------------------

\documentclass[11pt,a4paper,roman]{moderncv} % Font sizes: 10, 11, or 12; paper sizes: a4paper, letterpaper, a5paper, legalpaper, executivepaper or landscape; font families: sans or roman

%% for enies y tildes
\usepackage[utf8]{inputenc}
\usepackage[spanish]{babel}

\moderncvstyle % CV theme
\moderncvcolor % load the moderncvcolor.sty

%\usepackage{lipsum} % Used for inserting dummy 'Lorem ipsum' text into the template

\usepackage[scale=0.75]{geometry} % Reduce document margins
%\setlength{\hintscolumnwidth}{3cm} % Uncomment to change the width of the dates column
%\setlength{\makecvtitlenamewidth}{10cm} % For the 'classic' style, uncomment to adjust the width of the space allocated to your name

%----------------------------------------------------------------------------------------
%	NAME AND CONTACT INFORMATION SECTION
%----------------------------------------------------------------------------------------

\firstname{Leandro Ezequiel} % Your first name
\familyname{Nahabedian} % Your last name

% All information in this block is optional, comment out any lines you don't need
\title{Curriculum Vitae}
\address{Pumacahua 422}{Ciudad Autonoma de Buenos Aires, Argentina}
\mobile{+54 11 3779 5599}
%\phone{(000) 111 1112}
%\fax{(000) 111 1113}
\email{leanahabedian@gmail.com}
\homepage{nahabedian.herokuapp.com}{nahabedian.herokuapp.com} % The first argument is the url for the clickable link, the second argument is the url displayed in the template - this allows special characters to be displayed such as the tilde in this example
%\extrainfo{additional information}
\photo[70pt][0.2pt]{img/Lea.jpg} % The first bracket is the picture height, the second is the thickness of the frame around the picture (0pt for no frame)
%\quote{"A witty and playful quotation" - John Smith}

%----------------------------------------------------------------------------------------

\begin{document}

\makecvtitle % Print the CV title

%----------------------------------------------------------------------------------------
%	EDUCATION SECTION
%----------------------------------------------------------------------------------------

\section{Educación}

\cventry{2015--}{Estudiante de Doctorado en Ciencias de la Computación}{}{}{}{}
\cventry{2007--2014}{Licenciado en Ciencias de la Computación}{Universidad de Buenos Aires}{Argentina}{}{}%{\textit{GPA -- 8.0}}{First Class Honours}  % Arguments not required can be left empty
\cventry{2007--2013}{Analista Universitario en Computación}{Universidad de Buenos Aires}{Argentina}{}{}%{\textit{GPA -- 7.5}}{Specialized in }

\section{Tesis de Licenciatura}

\cvitem{Título}{\emph{Hot-Swap: Una técnica para la generación y actualización automática de controladores discretos en
tiempo de ejecución}}
\cvitem{Director}{Dr. Nicol\'as D'Ippolito}
\cvitem{Descripción}{Producir software que opera continuamente es un atributo de calidad muy común y posee muchas
aplicaciones. Por lo tanto, existe una necesidad de idear técnicas que pueden cambiar un sistema sin frenar o
interrumpir su ejecución cuando se produce un cambio en el ambiente y/o en los requerimientos.
\newline{}
En esta tesis, trabajamos el problema de realizar una actualización de controlador dinámicamente cuando la
especificación de un sistema cambia (tanto las asunciones del ambiente como los requerimientos).
\newline{}
Presentamos una solución general que, no solo produce un controlador que satisface la nueva especificación y maneja la
transición de uno al otro, sino que también, a diferencia de los trabajos ya existentes, forzamos al sistema actual a un
estado, el cual, dicha transición puede ocurrir. A su vez, usando síntesis de controladores mostramos como construir
automáticamente un controlador que garantiza que la actualización sucederá, y además, que dicha
actualización será de manera segura.
}

%----------------------------------------------------------------------------------------
%Jamee	WORK EXPERIENCE SECTION
%----------------------------------------------------------------------------------------

\section{Experiencia Academica}

\cventry{Mar 2014 Jun 2014}{Investigador asistente}{National Institute of Informatics}{Tokyo}{Japón}{
Investigación en ingenieria del software en síntesis de controladores con un efoque en software auto-adaptables.
Desarrollé una nueva técnica en la herramienta MTSA para construir controladores que permiten cambios en el ambiente y
asegura nuevos objetivos sin detener o interrumpir el sistema.
\newline{}%\newline{}
El trabajo realizado en mi tesis de Licenciatura se inicia en esta pasantía que fue llevada a cabo en Tokyo, Japón y
supervisado por el Dr. Nicolás D'Ippolito y el Dr. Kenji Tei. La herramienta MTSA fue desarrollada en \textsc{java} y
usted puede bajarlo del siguiente link \url{http://sourceforge.net/projects/mtsa}.
}

%===============================================

\section{Teaching Experience}

\cventry{Mar 2009 Nov 2009}{Tutor}{Freelance}{Buenos Aires}{Argentina}{
I gave tutorials to high school students. Tutorials were of mathematics, physics and English subjects. These students
attended to special schools to complete their studies.
}

%===============================================

\section{Industrial Experience}

\cventry{Nov 2012 Mar 2015}{Project Leader / Developer}{National Ministry of Interior}{Argentina}{}{
I coped with different tasks that requires programming in many languages. Mainly, I used to work with \textsc{python}
but also I worked with \textsc{java} and \textsc{php}. Usually I worked in back-end.
\newline{}
Then, I also work as a Project leader. We had to assign the time that each media outlet gave to each political party for
their advertising campaign for 2013 and 2015 Argentine elections. Our system also provide a web application that allow the media outlets and the
political parties to communicate between them, so as to decide which TV spot they will use and when.}

%------------------------------------------------
\cventry{May 2012 Oct 2012}{Python developer}{Core Security Technologies}{Buenos Aires}{Argentina}{
Company that develops software for analysis of vulnerabilities. My work consist in developing \textsc{python} code to
build a system that allowed to add, remove or edit tasks in \textsc{unix} crontab file.
\newline{}
Another project was to initiate a server in a virtual machine with \textsc{openBSD} to be used as a web server. To
achieve this goal I had to install all the packages needed to build a web service.
}

%------------------------------------------------
\cventry{May 2010 Dic 2010}{ActionScript2/Python developer}{MetroGames}{Buenos Aires}{Argentina}{
Company that makes video games for social networks. I was one of the developers of a particular game called Fashion
World. The design patter that we used to implement that game was \textsc{mvc} and the language was
\textsc{ActionScript2}.
\newline{}
Afterwards, I worked with \textsc{python} to make systems to be used within the company. These programs were part of the
internal infrastructure of the company. For instance, a system that manage the deployment of a project. 
}

%------------------------------------------------
\cventry{Feb 2010 May 2010}{Web developer}{Intelligenx}{Buenos Aires}{Argentina}{
I did an internship for this company where I had two projects.
\newline{}
First, I configured an application using \textsc{xpath} and \textsc{regular expressions}. By doing this, the tool fill a
database with information about stores of different companies.
\newline{}
After that, we produce a web page that used the data obtained before so as to provide yellow pages service. We used
\textsc{java} to create the application. 
}


%----------------------------------------------------------------------------------------
%	AWARDS SECTION
%----------------------------------------------------------------------------------------

\section{Honors \& Awards}

\cventry{Ago 2012 Ago 2014}{ICT Scholarship}{Ministry of Science, Technology and Productive Innovation}{Argentina}{}{
I got a scholarship for two years awarded by the Ministry of Science and Technology of the Republic of Argentina to
get the degree in master in computer science. The beneficiaries of this plan were chosen based on the number of test
taken, the years undertaken in the study and obtained grade point average.}

%----------------------------------------------------------------------------------------
%	LANGUAGES SECTION
%----------------------------------------------------------------------------------------

\section{Languages}

\cvitemwithcomment{Spanish}{Mothertongue}{}
\cvitemwithcomment{English}{Bilingual}{First Certificate in English in 2006 with C grade}

%----------------------------------------------------------------------------------------
%	INTERESTS SECTION
%----------------------------------------------------------------------------------------

%\section{Interests}

%\renewcommand{\listitemsymbol}{-~} % Changes the symbol used for lists

%\cvlistdoubleitem{Guitar}{Music}
%\cvlistdoubleitem{Cooking}{Dancing}
%\cvlistitem{Running}

\end{document}
