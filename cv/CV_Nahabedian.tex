\documentclass[10pt]{article}
% THIS IS FOR SPANISH
\usepackage[spanish]{babel}
\usepackage[utf8]{inputenc} 
\usepackage{url}
\usepackage{graphicx}
%% 
\usepackage[margin=3cm]{geometry}


% HEADER AND FOOTER
\usepackage{fancyhdr}
\pagestyle{fancyplain}
\lhead{Leandro Nahabedian}
\rhead{Curriculum Vitae}
\cfoot{Ciudad Autónoma de Buenos Aires - Argentina\\ DNI: 34.023.322 - +54 (911) 2386 2844 - 
lnahabedian@dc.uba.ar}
\renewcommand{\headrulewidth}{0.4pt}
\renewcommand{\footrulewidth}{0.4pt}
%%

\title{\bfseries\Huge Leandro Ezequiel Nahabedian}
\author{Curriculum Vitae}
\date{\today}

% COLUMNS DECLARATION
\usepackage{array, xcolor}
\definecolor{lightgray}{gray}{0.8}
\newcolumntype{L}{>{\raggedleft}p{0.12\textwidth}}
\newcolumntype{R}{p{0.8\textwidth}}
\newcommand\VRule{\color{lightgray}\vrule width 0.5pt}
%%

%\hyphenation{(CONICET)}
%\hyphenation{Argentina}


\begin{document}
\maketitle

\section*{Educación}
\begin{tabular}{L!{\VRule}R}
2015 -- & \textbf{Estudiante de doctorado en Ciencias de la Computación}, \textit{Universidad de Buenos Aires}, Argentina\\
2007--2014 & \textbf{Licenciado en Ciencias de la Computación}, \textit{Universidad de Buenos Aires}, Argentina\\
2007--2013 & \textbf{Analista Universitario en Computación}, \textit{Universidad de Buenos Aires}, Argentina\\

\end{tabular}

%----------------------------------------------------------------------------------------
%	AWARDS SECTION
%----------------------------------------------------------------------------------------

\section*{Premios y Becas}

\begin{tabular}{L!{\VRule}R}
	May 2016 & \textbf{Best Paper Award}, \textit{11th International Symposium 
	on Software Engineering for Adaptive and
		Self-Managing Systems}, Austin, TX, USA.\\
	& Mejor trabajo científico publicado en el congreso SEAMS 2016 de título:
	``Assured and correct dynamic update of controllers''\\
	
	
	Abr 2015 -- & \textbf{Beca doctoral}, \textit{Consejo Nacional de 
	Investigaciones Científicas y Técnicas (CONICET)}, Argentina\\
	
	%------------------------------------------------
	
	Ago 2012 Ago 2014 & \textbf{Beca TIC}, Ministerio de Ciencia, Tecnología e 
	Innovación Productiva, Argentina\\
	& \vspace{-0.7cm} Obtuve una beca por dos años, otorgada por el Ministerio 
	de Ciencia, Tecnología e Innovación Productiva de la República
	Argentina para obtener el título de Licenciado en Ciencias de la 
	Computación. Los beneficiarios de este plan fueron
	elegidos en base al número de exámenes rendidos, los años emprendidos en el 
	estudio y el promedio de notas obtenidas.\\
	
\end{tabular}

%----------------------------------------------------------------------------------------
%	WORK EXPERIENCE SECTION
%----------------------------------------------------------------------------------------

\section*{Experiencia Profesional Académica}

\subsubsection*{Publicaciones}

\begin{tabular}{L!{\VRule}R}
BPM 2019 & \textbf{L. Nahabedian}, V. Braberman, N D'Ippolito, J. Kramer, 
S. Uchitel.
\textbf{\textit{Dynamic Reconfiguration of Business Processes}}, en 
International Conference on Business Process Management 2019.\\	
IEEE-TSE 2018 & \textbf{L. Nahabedian}, V. Braberman, N D'Ippolito, S. Honiden, J. Kramer, K. Tei, 
S. Uchitel.
\textbf{\textit{Dynamic Update of Discrete Event Controllers}}, en Transactions on Software 
Engineering.\\
ICSE 2017 & \textbf{L.Nahabedian}. \textbf{Dynamic Update of Business Process 
Management}. en \textit{International Conference on Software Engineering - 
Doctoral Symposium}.\\
SEAMS 2016 & \textbf{L.Nahabedian}, V.Braberman, N.D'Ippolito, S.Honiden, 
J.Kramer, K.Tei, S.Uchitel. \textbf{Assured and Correct
Dynamic Controller Update} en \textit{International Symposium on Software 
Engineering for Adaptive and Self-Managing
Systems} - \includegraphics[scale=0.022]{../img/medal.png} 
\textbf{\textit{Best Paper Award}}
\end{tabular}

\subsubsection*{Formación de Recursos Humanos}

\begin{tabular}{L!{\VRule}R}
Mar 2015 Dic 2018 & \textbf{Director de Tesis de Licenciatura}, \textit{Universidad de Buenos 
Aires}, Argentina\\
& \vspace{-0.7cm} Dirijí a Victor Wjugow en su tesis de licenciatura: ``Consecutive Controller Hot 
Swaps: Mejora para la generación de múltiples controladores discretos y sus actualizaciones en 
tiempo de ejecución''.

El software que opera de manera continua necesita eventualmente ser actualizado evitando la baja de 
servicio por 
un tiempo indeterminado. Por lo tanto, hemos desarrollado tecnicas para no solo evitar este tiempo 
ocioso, sino también, para reemplazar a la ejecución vieja por una nueva sea cual fuere el estado 
actual.
La técnica desarrollada con Victor permite actualizaciones sucesivas suponinedo varios escenarios 
en los cuales el sistema debe estar funcionando.
Dependiendo de los objetivos que se quieran satisfacer y las capacidades que el sistema actual 
tiene, el software va evolucionando garantizando y utilizando los mismos.\\

%------------------------------------------------
Ene 2015 Jun 2017 & \textbf{Director de Tesis de Licenciatura}, \textit{Universidad de Buenos 
Aires}, Argentina\\
& \vspace{-0.7cm} Dirijí a Ivan Pasquini en su tesis de licenciatura: ``Localización y 
modelado simultáneos mediante generación y actualización automática de controladores discretos''.

El trabajo presenta una solución novedosa para problemas donde se desconoce la totalidad de las 
propiedades del ambiente. 
La estrategia es explorar dicho ambiente de una manera inteligente para obtener solamente la 
información necesaria para resolver el problema. 
Dicho trabajo tiene aplicaciones en la robótica, en las redes de comunicación y otras.\\
\end{tabular}

\begin{tabular}{L!{\VRule}R}
%------------------------------------------------
Jul 2017 & \textbf{Jurado de Tesis de Licenciatura}, \textit{Universidad de Buenos 
Aires}, Argentina\\
& Jurado de tesis de licenciatura de título: ``Hacia un entorno declarativo para la 
especificación temprana de comportamiento de Arquitecturas de Software'', 
presentada por Francisco Tarulla.
\end{tabular}

\subsubsection*{Participación en Proyectos}

\begin{tabular}{L!{\VRule}R}
Mar 2014 Jun 2014 & \textbf{Investigador asistente}, \textit{National Institute of Informatics}, Tokyo, Japón\\
& \vspace{-0.7cm} Investigación en ingeniería del software en síntesis de controladores con un enfoque en software auto-adaptables.
Desarrollé una nueva técnica en la herramienta MTSA para construir controladores que permiten cambios en el ambiente y
asegura nuevos objetivos sin detener o interrumpir el sistema.

El trabajo realizado en mi tesis de Licenciatura se inicia en esta pasantía que fue llevada a cabo en Tokyo, Japón y
supervisado por el Dr. Nicolás D'Ippolito y el Dr. Kenji Tei. La herramienta MTSA fue desarrollada en \textsc{java} y
usted puede bajarlo del siguiente link \url{http://mtsa.dc.uba.ar}.\\
\end{tabular}

%===============================================

\section*{Experiencia Docente}

\begin{tabular}{L!{\VRule}R}
Ago 2019 -- & \textbf{Jefe de Trabajos Prácticos}, \textit{Universidad de 
	Buenos 	Aires}, Argentina\\
& Jefe de trabajos prácticos de Ingeniería del Software II.\\		
Ago 2017 Jul 2019 & \textbf{Ayudante de Primera}, \textit{Universidad de 
	Buenos 	Aires}, Argentina\\
& \vspace{-0.7cm} Ayudante de primera de Ingeniería del Software I y II.\\		
\vspace{-0.7cm} Mar 2017 Jul 2017 & \vspace{-0.7cm} \textbf{Jefe de Trabajos Prácticos}, 
\textit{Universidad de 
Buenos 	Aires}, Argentina\\
& \vspace{-0.7cm} Jefe de trabajos prácticos de Ingeniería del Software I.\\	
\vspace{-0.7cm}Ago 2015 Feb 2017 & \vspace{-0.7cm}\textbf{Ayudante de Primera}, \textit{Universidad 
de Buenos 
Aires}, Argentina\\
& \vspace{-0.7cm} Ayudante de primera de Ingeniería del Software I y II.\\
\vspace{-0.7cm}Mar 2009 Nov 2009 & \vspace{-0.7cm}\textbf{Clases de apoyo}, 
\textit{Freelance}, Argentina\\
& \vspace{-0.7cm} Dictado de clases particulares a estudiantes de colegio secundario. Dichas clases eran de matemática, física e inglés.
Estos estudiantes concurrían a colegios especiales para completar sus estudios secundarios.\\
\end{tabular}


%===============================================

\section*{Experiencia Profesional en la Industria}

\begin{tabular}{L!{\VRule}R}
Nov 2012 Abr 2015 & \textbf{Líder de proyecto / Desarrollador}, \textit{Ministerio del Interior}, Argentina\\
& \vspace{-0.7cm} Me desenvolví en diferentes tareas que requerían programar en varios lenguajes de programación. Principalmente,
trabajaba con \textsc{python} pero también trabajé con \textsc{java} y \textsc{php}. Mi actividad era básicamente el
desarrollo en back-end.

Luego, trabajé también como Líder de Proyecto. El proyecto que llevé a cabo consistía en asignar el tiempo que los
medios de comunicación ceden a las agrupaciones políticas para campañas electorales del 2013 y el 2015. Nuestro sistema
contaba con una aplicación web que permitía a los medios de comunicación y a las agrupaciones políticas comunicarse
entre ellos, para poder decidir cuales son los spots publicitarios que se utilizarán durante la campaña en cada medio.\\


%------------------------------------------------
May 2012 Oct 2012 & \textbf{Desarrollador Python}, \textit{Core Security Technologies}, Buenos Aires, Argentina\\
& \vspace{-0.7cm} Compañía que desarrolla software para análisis de vulnerabilidades. Mi trabajo consistió en desarrollar código
\textsc{python} para construir un sistema que permite agregar, remover o editar tareas en el archivo crontab de
\textsc{unix}.

Otro proyecto en el cual estuve involucrado fue inicializar un servidor en una máquina virtual con \textsc{openBSD} para
ser usado como servidor web. Para lograr dicho objetivo tuve que instalar todos los paquetes necesario para construir un
servicio web.\\


%------------------------------------------------
May 2010 Dic 2011 & \textbf{Desarrollador ActionScript2/Python}, \textit{MetroGames}, Buenos Aires, Argentina\\
& \vspace{-0.7cm} Compañía que desarrolla video juegos para redes sociales. Fui uno de los desarrolladores de un juego particular llamado
Fashion World. Utilizábamos el patrón de diseño \textsc{mvc} y bajo el lenguaje 
\textsc{ActionScript2}.

Luego, trabajé con \textsc{python} para realizar sistemas para uso 
dentro de la compañía. Estos programas
formaban parte de la infraestructura interna de dicha empresa.\\


%------------------------------------------------
Feb 2010 May 2010 & \textbf{Desarrollador Web}, \textit{Intelligenx}, Buenos Aires, Argentina\\
& \vspace{-0.7cm} Realicé una pasantía para esta compañía donde tuve dos proyectos.
El primero fue configurar una aplicación web usando \textsc{xpath} y 
\textsc{regular expressions}.
Luego de esto, producimos una página web que usaba estos datos obtenidos para proveer un servicio de páginas amarillas.
Usamos \textsc{java} para crear la aplicación.\\
\end{tabular}


%----------------------------------------------------------------------------------------
%	LANGUAGES SECTION
%----------------------------------------------------------------------------------------

\section*{Idiomas}

\begin{tabular}{L!{\VRule}R}
{\bf Español}&{\bf Nativo}\\
{\bf Inglés} &{\bf B2} (First Certificate in English 2006)\\
\end{tabular}


\end{document}


