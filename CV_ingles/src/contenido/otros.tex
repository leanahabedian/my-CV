\section{Otros Conocimientos}

\subsection{Programación}

\hspace{0.4cm}
Al estudiar ciencias de la computación poseo un dominio amplio de la programación. Los 
lenguajes que domino ampliamente son Python, Ruby, C, C++, Java y assembler, aunque mi 
carrera provee facilidad de adaptación a nuevos lenguajes. Experiencia en el uso de xpath y 
expresiones regulares. Practiqué de manera academica el funcionamiento de metodologias ágiles como
\textbf{Scrum}, \textbf{Extreme Programing}, entre otras.

Además, poseo otros conocimientos que facilitan la programación como la del uso de sistema de control de versiones como \textbf{SVN}, \textbf{Mercurial} y \textbf{Git}.

Por otra parte, obtengo gran conocimiento de la programación de bajo nivel y un fuerte estudio de la estructura interna de una computadora de tecnología IA­32 (Intel Arquitecture de 32bits).

Por último, destaco que suelo darle importancia a la buena presentación de mis textos y/o publicaciones, por lo que recomiendo fuertemente y uso $\LaTeX$ para realizarlas. Un claro ejemplo es este Currículum Vitae que usted esta leyendo.


\subsection{Configuración de Sistemas Operativos}

\hspace{0.4cm}
Muchas de las materias de mi carrera tienen como objeto el funcionamiento de los sistemas 
operativos, estudiando por separado cada componente, como el Memory Management Unit (MMU), I/O unit, scheduler, etc. y su funcionamiento en sistemas distribuidos utilizando algoritmos conocidos. Cuento con experiencia de programación con threads o multi-procesos, utilizando IPC (inter-process communication), locking, entre otras técnicas para manejar problemas que surgen por usar muchos hilos de ejecucion.

Otra área la cual me interesa mucho y relacionada con redes, es el area de seguridad informática, si bien no tengo
mucha experiencia laboral en esto, fueron conceptos que siempre me llamaron la atención. Finalizando, me gustaría agregar que también soy usuario domestico del sistema operativo Linux distribución Ubuntu desde el 2009.
