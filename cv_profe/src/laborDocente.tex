Son muchos los t\'opicos del \'area de Ingenier\'ia de Software que se relacionan con mis
temas de investigaci\'on y docente, en los cuales, estoy totalmente capacitado.
Esta afirmaci\'on puede ser comprobada con mi historial docente por un lado, y con los peque\~nos logros producidos en el \'area de la ciencia.

Dichos conceptos que impartir\'e en caso de obtener el cargo en cuesti\'on, forman parte de
la curricula de varias signaturas de la carrera de ciencia de la computaci\'on de la universidad. Como por ejemplo, Ingenier\'ia del Software 1 y 2, Base de datos y Teoria de lenguajes, entre otras.

Como docente a cargo de alguna de estas signaturas, creo de suma importancia poder
guiar a mis alumnos a que puedan explorar la terminolog\'ia correcta, buscando potenciar 
las distintas cualidades y habilidades que los alumnos posean. Todo lo aprendido
ser\'a evaluado para obtener una prueba del conocimiento adquirido por los alumnos.
La finalidad de evaluaci\'on es generar el perfil deseado del alumno al final de la cursada. Todo este trabajo con los alumnos ser\'a efectuado en conjunto con los 
docentes ayudantes, quienes colaboraran con la evaluaci\'on de los trabajos pr\'acticos.

Por otro lado, creo que los programas de las materias actuales pueden ser levemente
modificadas, ya que la computaci\'on se encuentra en continuo movimento y la conexi\'on
entre los problemas cl\'asicos y los m\'as recientes es necesario.
Sin alterar los programas de las diferentes curriculas pretendo extender el
conocimiento de los alumnos con problemas actuales para que tengan conocimiento
de los problemas modernos.
  

%Varios temas del \'area de Ingenier\'ia de Software forman parte de
%la curr\'icula de Cs. de la Computaci\'on de la ACM, as\'i como de
%los programas de las carreras de Cs. de la Computaci\'on en
%universidades de todo el mundo.

%Dentro de las materias de la Licenciatura en Ciencias de
%la Computaci\'on de la FCEyN, UBA, estos temas se abordan en materias como 
%Bases de Datos, Ingenier\'ia de Software 1 y 2,
%Administraci\'on de proyectos inform\'aticos y An\'lisis de Requerimientos
%Temporales entre otras.


%Teniendo en cuenta mi historial docente y siendo egresado del actual plan de 
%estudios, como Profesor Adjunto en el Departamento de Computaci\'on estoy en 
%condiciones de dictar cualquier materia que el Departamento considere 
%necesario. 

%Sin embargo, dado que presente el
%concurso es espec\'ificamente para dictar la materia Inger\'ia de
%Software 1, me concentrar\'e en dar mi visi\'on y propuesta para la misma. 

%Si bien Ingenier\'ia de Software I incluye hoy en su programa una
%extensa introducci\'on a la Ingenier\'ia de Software, considero que
%en la materia hay lugar para presentar tanto los problemas cl\'asicos
%del area, como tambi\'en un panorama sobre distintas t\'ecnicas y herramientas 
%que
%hacen al estado del arte de la Ingenier\'ia de Software. 
%Con esto en mente, concetrar\'ia mis esfuerzos en no solo mantener el gran 
%nivel que la materia tiene hoy, sino en avanzar sobre algunas de estas 
%t\'ecnicas y herramientas tienen especial potencial para la interacci\'on
%cient\'ifico tecnol\'ogica. 
%Para esto, me propongo profundizar en la
%materia el uso de t\'ecnicas de an\'alisis, relevamiento y
%validaci\'on de requerimientos provenientes de la ingenier\'ia de
%requerimientos, y t\'ecnicas de
%validaci\'on y verficaci\'on basadas en modelos de comportamiento
%como por ejemplo Model Checking. Adem\'as presentar\'ia un panorama
%sobre t\'ecnicas de generaci\'on autom\'atica de casos de prueba en el marco
%t\'ecnicas de testing automatizado. 
%
%Cabe destacar que los cambios que propongo en la materia est\'an 
%motivados no solo en mi experiencia como investgador en el \'area de Ingenier\'ia 
%de Software, sino tambi\'en en mi conocimiento de la materia dado 
%por mi trabajo como docente en la misma. Los alumnos de la materia son 
%estudiantes jovenes y en su mayoria trabajadores en la industria del software. 
%En este contexto, motivarlos a los alumnos a la aplicaci\'on de dichas t\'ecnicas 
%y herramientas 
%es clave para 
%aumentar la adopci\'on en la industria de propuestas con tanto potencial como la 
%generaci\'on autom\'atica de casos de prueba. Todo esto con miras a mejorar la 
%calidad del software que se produce en el pais y, porque no, incrementalmente 
%mejorar la posici\'on del pais en cuanto a la industria del software a nivel 
%mundial. Esto \'ultimo es de 
%gran 
%importancia en un 
%contexto como el argentino en el cual muchos sectores declarados de relavancia 
%nacional, como por ejemplo AgroTICs, Energ\'ia, y Transporte, podr\'an 
%beneficiarse con la adopci\'on de nuevas t\'ecnicas de elaboraci\'on de 
%requeriminetos, verificaci\'on de software, y testing automatizado. 
%
%Por \'ultimo, en cuanto a materias optativas, considero muy importante la 
%ampliaci\'on y actualizaci\'on de la oferta actual, especialmente en el \'area 
%docente de Teor\'ia. En particular, desde mi experiencia planeo proponer una 
%materia 
%optativa basada en mis temas de investigaci\'on, ya sea en t\'ecnicas de control 
%autom\'atico o bien en la aplicaci\'on de dichas a la rob\'otica, \'area en la cual 
%estoy trabajando 
%con un grupo de alumnos del departamento.  
%
%
%
%%%%%%%%%%%%%%%
%Para las materias optativas propongo utilizar principalmente la
%modalidad de seminario, alternando clases te\'ericas con exposici\'on
%de temas por parte de los alumnos. El objetivo es profundizar en
%ciertos problemas de inter\'es actual, que variar\'{\i}an en cada
%cuatrimestre en que se dicte la materia, y poniendo especial
%\'enfasis en las preguntas abiertas alrededor de dichos problemas.
%Se espera que el seminario despierte el inter\'es de los alumnos,
%dando lugar a nuevas Tesis de Licenciatura y Doctorado en el \'area
%y la incorporaci\'on de los mismos al grupo de investigaci\'on.
