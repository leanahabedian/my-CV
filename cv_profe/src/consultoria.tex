%%agregar internet de la ciudad

\item \textbf{Noviembre 2007 / en curso:} Planificaci\'on de la
recolecci\'on de residuos en contenedores en la zona sur de la
Ciudad de Buenos Aires.

Este proyecto se enmarca en un convenio entre la Facultad de
Ciencias Exactas y Naturales (FCEN, UBA) y el Ente de Higiene
Urbana (EHU) del Gobierno de la Ciudad de Buenos Aires, y tiene
como objetivo estudiar el problema de recolecci\'on de
contenedores de residuos domiciliarios en una de las zonas en las
que se encuentra dividida la recolecci\'on de residuos en la
ciudad, que est\'a a cargo del gobierno de la ciudad. El EHU
cuenta con una flota de camiones de recolecci\'on de contenedores,
que recorren en dos turnos el \'area de cobertura. Este trabajo
consiste en estudiar modelos y t\'ecnicas de zonificaci\'on del
\'area de cobertura, y proponer algoritmos de ruteo de cada
cami\'on. Adem\'as de la colaboraci\'on entre la FCEN y el EHU,
este proyecto deriva en la realizaci\'on de dos tesis de
licenciatura en ciencias matem\'aticas y ciencias de la
computaci\'on, respectivamente.

\item \textbf{Junio 2007 / Julio 2008:} Dise\~no del fixture de la
liga de primera divisi\'on de v\'oley masculino.

La liga de v\'oley masculino de primera divisi\'on de Argentina
est\'a conformada por 12 equipos y consta de una fase regular
seguida de playoffs. En la fase regular se enfrentan todos los
equipos entre s\'{\i}, en condici\'on de local y visitante. Una
caracter\'{\i}stica interesante de esta liga es que los equipos se
agrupan en parejas, que se enfrentan entre s\'{\i} en pares de
fechas consecutivas. Este proyecto para la Asociaci\'on de Clubes
Liga Argentina de V\'oleibol (ACLAV) consisti\'o en la
optimizaci\'on del fixture para minimizar las distancias totales
de viaje de los equipos en la liga 2007/2008, teniendo en cuenta
las restricciones de local\'{\i}a y condiciones adicionales de
equidad deportiva. El fixture se utiliz\'o satisfactoriamente
durante la liga que concluy\'o en abril de 2008.


\item \textbf{Junio 2005 / Marzo 2006:} Desarrollo de una
herramienta de planificaci\'on para la circulaci\'on de trenes de
Am\'erica Latina Log\'{\i}stica (ALL), utilizando t\'ecnicas de
optimizaci\'on combinatoria.

ALL cuenta con 15.000 kil\'ometros de l\'{\i}neas f\'erreas en
Brasil y Argentina, m\'as de 550 locomotoras y 17.000 vagones. ALL
Argentina transporta por a\~no casi 5 millones de toneladas de
diferentes tipos de mercader\'{\i}a, entre ellos contenedores,
cereales, minerales y productos de consumo. El objetivo del
optimizador es planificar cada d\'{\i}a la circulaci\'on de trenes
teniendo en cuenta un horizonte de planificaci\'on de 4 d\'{\i}as.
En base a las demandas de cada tipo de mercader\'{\i}a, la
disponibilidad de vagones y locomotoras en cada patio, la
capacidad de operaci\'on en cada patio, y las restricciones
propias de todos los elementos involucrados, se buscar\'a una
asignaci\'on de demandas a trenes que minimice los costos de
operaci\'on y maximice la facturaci\'on proveniente de las
demandas atendidas. Este proyecto se desarroll\'o bajo la
modalidad de orden de asistencia t\'ecnica ALL--FCEN, y fue
realizado en conjunto con GAPSO Servi\c{c}os de Inform\'atica
Ltda., Rio de Janeiro, Brasil.

\item \textbf{2004:} Elaboraci\'on, toma y correcci\'on de una
prueba de conocimientos de programaci\'on en el marco de un
proceso de selecci\'on de personal, convenio AFIP--FCEyN.


\item \textbf{1999 y 2001:} Correcci\'on de evaluaciones de
calidad educativa en la educaci\'on secundaria, \'area
matem\'atica, convenio Ministerio de Educaci\'on--FCEyN.
