\documentclass[10pt]{article}
% THIS IS FOR SPANISH
\usepackage[spanish]{babel}
\usepackage[utf8]{inputenc} 
%% 
\usepackage[margin=3cm]{geometry}
\usepackage{url}


% HEADER AND FOOTER
\usepackage{fancyhdr}
\pagestyle{fancyplain}
\lhead{Leandro Nahabedian}
\rhead{Curriculum Vitae}
\cfoot{Ciudad Autónoma de Buenos Aires - Argentina\\DNI: 34.023.322 - +54 11 3779 5599 - lnahabedian@dc.uba.ar}
\rfoot{\thepage}
\renewcommand{\headrulewidth}{0.4pt}
\renewcommand{\footrulewidth}{0.4pt}
%%

\title{\bfseries\Huge Leandro Ezequiel Nahabedian}
\author{Curriculum Vitae}
\date{\today}

% COLUMNS DECLARATION
\usepackage{array, xcolor}
\definecolor{lightgray}{gray}{0.8}
\newcolumntype{L}{>{\raggedleft}p{0.12\textwidth}}
\newcolumntype{R}{p{0.8\textwidth}}
\newcommand\VRule{\color{lightgray}\vrule width 0.5pt}
%%

%\hyphenation{(CONICET)}
%\hyphenation{Argentina}
%\hyphenation{MTSA}


\begin{document}
\maketitle


\section{Antecendentes Docentes}

\subsection*{Universitarios}

\begin{tabular}{L!{\VRule}R}
Mar 2017 Jun 2017 & \textbf{Jefe de Trabajos Prácticos}, \textit{Universidad de 
Buenos Aires}, Argentina\\
& \vspace{-0.7cm} Jefe de Trabajos Prácticos interino de Ingeniería del 
Software II en la Universidad de Buenos Aires.\\

Mar 2016 Jun 2017 & \textbf{Ayudante de primera}, \textit{Universidad de Buenos 
Aires}, 
Buenos Aires, Argentina\\
& \vspace{-0.7cm} Ayudante de primera interino de Ingeniería del Software I en 
la Universidad de Buenos Aires.\\

Ago 2015 Feb 2016 & \textbf{Ayudante de primera}, \textit{Universidad de Buenos 
Aires}, Argentina\\
& \vspace{-0.7cm} Ayudante de primera interino de Ingeniería del Software I en 
la Universidad de Buenos Aires.\\

\end{tabular}

\subsection*{Otras Actividades Docentes}

\begin{tabular}{L!{\VRule}R}
Mar 2009 Nov 2009 & \textbf{Clases de apoyo}, \textit{Freelance}, Buenos Aires, Argentina\\
& \vspace{-0.7cm} Dictado de clases particulares a estudiantes de colegio secundario. Dichas clases eran de matemática, física e inglés.
Estos estudiantes concurrían a colegios especiales para completar sus estudios secundarios.\\
\end{tabular}

\newpage

\section{Antecendentes Cientificos}

\subsection*{Trabajos Publicados}

\begin{tabular}{L!{\VRule}R}
ICSE 2017 & \textbf{L. Nahabedian}.
\textbf{\textit{Dynamic Update of Business Process Management}}, in 39th 
International Conference on Software Engineering - Doctoral Symposium - 
\textit{Buenos Aires, Argentina.}\\
SEAMS 2016 & \textbf{L. Nahabedian}, V. Braberman, N D'Ippolito, S. Honiden, J. Kramer, K. Tei, S. Uchitel.
\textbf{\textit{Assured and Correct Dynamic Controller Update}} in 11th International Symposium on Software Engineering
for Adaptive and Self-Managing Systems - \textit{Austin, Texas, USA.} \textit{\textbf{Best paper award}}\\
JAIIO/ASSE 2016 & \textbf{L. Nahabedian}, V. Braberman, N D'Ippolito, S. Honiden, J. Kramer, K. Tei, S. Uchitel.
\textbf{\textit{Assured and Correct Dynamic Controller Update}} en Jornadas Argentinas de 
Informática / Simposio Argentino de Ingeniería de Software (Oral Communication) - \textit{Buenos 
Aires, Argentina}\\
\end{tabular}

\subsection*{Trabajos Pendintes de Evaluación}

\begin{tabular}{L!{\VRule}R}
ACM-TAAS & \textbf{L. Nahabedian}, V. Braberman, N D'Ippolito, S. Honiden, J. Kramer, K. Tei, S. 
Uchitel.
\textbf{\textit{Discrete Event Controller Dynamic Update}} en ACM Transaction 
on Autonomous 
and Adaptive Systems. (Journal paper).\\
\end{tabular}

\subsection*{Participación en Congresos Internacionales/Nacionales}

\subsubsection*{Asistencia a Congresos}

\begin{tabular}{L!{\VRule}R}
Sep 2016 & \textbf{PRE-ICSE 2017}, \textit{Estudiante aceptado}\\
&Evento donde los estudiantes presentaban un abstract de su tema de investigacion para ser 
aceptados. En el simposio tres investigadores
formados nos dieron feedback de nuestros temas, además de charlas introductorias a la conferencia 
ICSE 2017.\\
Sep 2016 & \textbf{JAIIO/ASSE 2016}, \textit{Presentador}\\
& Presenté la Oral communication en Jornadas Argentinas de Informática / Simposio Argentino de 
Ingeniería de Software, Buenos Aires, Argentina.\\
May 2016 & \textbf{SEAMS 2016}, \textit{Presentador}\\
& Presenté el paper aceptado en la conferencia 11th International Symposium on 
Software Engineering 
for Adaptive and Self-Managing Systems - Austin, Texas, USA.\\
May 2016 & \textbf{ICSE 2016}, \textit{Estudiante Voluntario}\\
&Estudiante voluntario en International Conference on Software Engineering - Austin, Texas, USA. 
Como voluntario pude participar de las charlas y discutir ideas con los disertantes de dicha 
conferencia.\\
Ago 2015 & \textbf{IJCAI 2015}, \textit{Estudiante Voluntario}\\
&Estudiante voluntario en International Joint Conference on Artificial Intelligence, Buenos Aires, 
Argentina, 2015. Como
voluntario pude participar de las charlas y discutir ideas con los disertantes de dicha 
conferencia.\\
\end{tabular}

\subsubsection*{Organización de Congresos Internacionales}

\begin{tabular}{L!{\VRule}R}
Oct 2016 -- & \textbf{ICSE 2017} \textit{Web Chair}\\
& Chair de la página web de International Conference on Software Engineering. 
Esto significa 
comunicarse con los
chairs de los diferentes tracks/eventos de la conferencia para cargar los datos 
en la pagina web: \url{http://icse2017.gatech.edu/}.\\
\end{tabular}


\subsubsection*{Otras Participaciones}


\begin{tabular}{L!{\VRule}R}
Jun 2015 & \textbf{ICTAC 2015}, \textit{Research Papers Subreviewer}\\
& Evalué research papers presentados en International Colloquium on theoretical aspects of 
computing 2015, Cali,
Colombia, colaborando con Nicolás D'Ippolito, miembro del Program Committee de dicha conferencia.\\
May 2015 & \textbf{ESEC/FSE 2015}, \textit{Tool Demonstration Subreviewer}\\
& Evalué papers de demostraciones de herramientas presentadas en Foundations of Software 
Engineering 2015, Bergamo,
Italia colaborando con Nicolás D'Ippolito, miembro del Program Committee de dicha conferencia.\\
May 2015 & \textbf{ESEC/FSE 2015}, \textit{Replication Package Evaluation Subreviewer}\\
& Evalué los paquetes de replicación de resultados presentados en Foundations of Software 
Engineering 2015, Bergamo,
Italia, colaborando con Nicolás D'Ippolito, miembro del Program Committee de dicha conferencia.\\
Ene 2015 & \textbf{FM 2015}, \textit{Research Papers Subreviewer}\\
& Evalué research papers presentados en Formal Methods 2015, Oslo, Noruega, colaborando con Diego 
Garbervetsky, miembro
del Program Committee de dicha conferencia.\\
Nov 2014 & \textbf{FSE 2014}, \textit{Research Paper Subreviewer}\\
& Evalué research papers presentados en Foundations of Software Engineering 2014, Hong Kong, 
colaborando
con Victor Braberman, miembro del Program Committee de dicha conferencia.\\
\end{tabular}

\subsection*{Formación de Recursos Humanos}

\begin{tabular}{L!{\VRule}R}

Mar 2015 -- & \textbf{Director de Tesis de Licenciatura}, \textit{Universidad de Buenos Aires}, Argentina\\
& Dirijo a Victor Wjugow, quien realizó una pasantía en el National Institute of Informatics, Japón. Trabajando
en conjunto estamos desarrollando una extensión al trabajo realizado durante mi estadía en dicho instituto.

Controladores que pueden soportar actualizaciones en diferentes capas de abstracción son producidos por la técnica que
estamos desarrollando. Estos controladores pueden ser actualizados cuando el ambiente en el que se ejecuta cambia,
los objetivos a satisfacer son imposible de garantizar o adquiero una capacidad nueva y debo aprovecharla. Cómo
soportar estas actualizaciones multi-capa es una pregunta común en el área de 
software auto adaptable. (Estado: iterando la escritura de su tesis)\\

%------------------------------------------------
Ene 2015 -- & \textbf{Director de Tesis de Licenciatura}, \textit{Universidad de Buenos Aires}, Argentina\\
& Dirijo a Ivan Pasquini, quien se está involucrando en resolver problemas de síntesis de controladores con
observación parcial para llevar a cabo esta tesis. Tenemos esperado finalizar pronto dicha tesis debido
a que Ivan ya se encuentra escribiendo su tesis.

El trabajo presentará una solución novedosa para problemas donde se desconoce la totalidad de las propiedades del
ambiente. La estrategia es explorar dicho ambiente de una manera inteligente para obtener solamente la información
necesaria para resolver el problema. Dicho trabajo tiene aplicaciones en la robótica, en las redes de comunicación y
otras. (Estado: Busqueda de Jurado evaluador)\\


\end{tabular}

\subsection*{Participación en Proyectos}

\begin{tabular}{L!{\VRule}R}
Mar 2014 Jun 2014 & \textbf{Investigador asistente}, \textit{National Institute of Informatics}, Tokyo, Japón\\
& \vspace{-0.7cm} Investigación en ingeniería del software en síntesis de controladores con un enfoque en software auto-adaptables.
Desarrollé una nueva técnica en la herramienta MTSA para construir controladores que permiten cambios en el ambiente y
asegura nuevos objetivos sin detener o interrumpir el sistema.

El trabajo realizado en mi tesis de Licenciatura se inicia en esta pasantía que fue llevada a cabo en Tokyo, Japón y
supervisado por el Dr. Nicolás D'Ippolito y el Dr. Kenji Tei. La herramienta MTSA fue desarrollada en \textsc{java} y
usted puede bajarlo del siguiente link \url{http://mtsa.dc.uba.ar}.\\
\end{tabular}

\subsection*{Participación en Workshops}

\begin{tabular}{L!{\VRule}R}
Mar 2017 & \textbf{FACAS 2017}, \textit{Participación}\\
& Participé en el Argentinian Workshop on Foundations for Automatic 
Construction and Analysis of
Software (FACAS) del 2 al 4 de marzo de 2017.\\
Mar 2016 & \textbf{FACAS 2016}, \textit{Participación}\\
& Participé en el Argentinian Workshop on Foundations for Automatic Construction and Analysis of
Software (FACAS) del 3 al 5 de marzo de 2016.\\
Mar 2015 & \textbf{FACAS 2015}, \textit{Disertante}\\
& Participé como disertante en el Argentinian Workshop on Foundations for Automatic Construction and Analysis of
Software (FACAS) del 17 al 19 de marzo de 2015.\\
Nov 2014 & \textbf{NII-UBA Workshop}, \textit{Disertante}\\
& Participé como disertante en el 2do workshop entre los investigadores del National Institute of Informatics, Japón y
de la Universidad de Buenos Aires. 
\end{tabular}






\newpage

\section{Antecedentes de Extensión}

\subsection*{Actividades}

\begin{tabular}{L!{\VRule}R}
Jun 2015 & \textbf{Organizador}, \textit{Semana de la Computación 2015}, FCEN, UBA\\
& Durante la semana de la computación me desenvolví como divulgador/organizador del evento que sucedió el 16, 17 y 18 de
junio del 2015. Me encargué de
agendar el orden de los colegios en cada taller y de guiarlos por dentro de la facultad. Además organicé el almuerzo de
cierre del evento, donde todos los divulgadores y organizadores son agasajados por el trabajo realizado.\\
Jun 2017 & \textbf{Charlista}, \textit{Semana de la Computación 2017}, FCEN, UBA\\
& Fui orador de la charla ``Yo apreté el botón apagar. Nunca dije actualizar'', que sucedió en la Semana de la Computación el 13, 14 y 15 de junio de 2018.

\end{tabular}

\newpage

\section{Antecedentes Profesionales}

\subsection*{Actividades profesionales fuera del ámbito académico}

\begin{tabular}{L!{\VRule}R}
Nov 2012 Abr 2015 & \textbf{Líder de proyecto / Desarrollador}, \textit{Ministerio del Interior}, Argentina\\
& \vspace{-0.7cm} Me desenvolví en diferentes tareas que requerían programar en varios lenguajes de programación. Principalmente,
trabajaba con \textsc{python} pero también trabajé con \textsc{java} y \textsc{php}. Mi actividad era básicamente el
desarrollo en back-end.

Luego, trabajé también como Líder de Proyecto teniendo 2 personas a cargo. El proyecto que llevé a cabo consistía en
asignar el tiempo que los medios de comunicación ceden a las agrupaciones políticas para campañas electorales del 2013 y
el 2015. Nuestro sistema contaba con una aplicación web que permitía a los medios de comunicación y a las agrupaciones
políticas comunicarse entre ellos, para poder decidir cuales son los spots publicitarios que se utilizarán durante la
campaña en cada medio.\\


%------------------------------------------------
May 2012 Oct 2012 & \textbf{Desarrollador Python}, \textit{Core Security Technologies}, Buenos Aires, Argentina\\
& \vspace{-0.7cm} Compañía que desarrolla software para análisis de vulnerabilidades. Mi trabajo consistió en desarrollar código
\textsc{python} para construir un sistema que permite agregar, remover o editar tareas en el archivo crontab de
\textsc{unix}.

Otro proyecto en el cual estuve involucrado fue inicializar un servidor en una máquina virtual con \textsc{openBSD} para
ser usado como servidor web. Para lograr dicho objetivo tuve que instalar todos los paquetes necesario para construir un
servicio web.\\


%------------------------------------------------
May 2010 Dic 2011 & \textbf{Desarrollador ActionScript2/Python}, \textit{MetroGames}, Buenos Aires, Argentina\\
& \vspace{-0.7cm} Compañía que desarrolla video juegos para redes sociales. Fui uno de los desarrolladores de un juego particular llamado
Fashion World. El patrón de diseño que utilizábamos para implementar ese juego fue \textsc{mvc} y el lenguaje utilizado
fue\textsc{ActionScript2}.

Además, también trabajé con \textsc{python} para realizar sistemas para uso dentro de la compañía. Estos programas
formaban parte de la infraestructura interna de dicha empresa. Un ejemplo de estos, es el sistema que se encargaba de
hacer el ``deployment'' de un proyecto.\\


%------------------------------------------------
Feb 2010 May 2010 & \textbf{Desarrollador Web}, \textit{Intelligenx}, Buenos Aires, Argentina\\
& \vspace{-0.7cm} Realicé una pasantía para esta compañía donde tuve dos proyectos.
El primero fue configurar una aplicación web usando \textsc{xpath} y \textsc{regular expressions}. Realizando dicha
configuración, la herramienta almacenaba en una base de datos información de locales de diferentes compañías situadas en
EEUU y Polonia.
Luego de esto, producimos una página web que usaba estos datos obtenidos para proveer un servicio de páginas amarillas.
Usamos \textsc{java} para crear la aplicación.\\
\end{tabular}

\newpage

\section{Calificaciones, Títulos, Estudios y Otros}

\subsection*{Títulos Obtenidos}
\begin{tabular}{L!{\VRule}R}
2015 -- & \textbf{Estudiante de doctorado en Ciencias de la Computación}, \textit{Universidad de Buenos Aires}, Argentina\\
2007--2014 & \textbf{Licenciado en Ciencias de la Computación}, \textit{Universidad de Buenos Aires}, Argentina\\
2007--2013 & \textbf{Analista Universitario en Computación}, \textit{Universidad de Buenos Aires}, Argentina\\
2004--2006 & \textbf{First Certificate in English}, \textit{University of Cambridge}, UK\\

\end{tabular}

\subsubsection*{Tesis de Licenciatura}

\begin{tabular}{L!{\VRule}R}
Título & \textbf{Hot-Swap: Una técnica para la generación y actualización automática de controladores discretos en
tiempo de ejecución}\\
Director & \textit{Dr. Nicolás D'Ippolito}\\
Descripción & Existe una necesidad de idear técnicas que pueden cambiar un sistema sin frenar o
interrumpir su ejecución cuando se produce un cambio en el ambiente y/o en los requerimientos.

En esta tesis, trabajamos el problema de realizar una actualización de controlador dinámicamente cuando la
especificación de un sistema cambia (tanto las asunciones del ambiente como los requerimientos).

Presentamos una solución general que, no solo produce un controlador que satisface la nueva especificación y maneja la
transición de uno al otro, sino que también, a diferencia de los trabajos ya existentes, forzamos al sistema actual a un
estado, el cual, dicha transición puede ocurrir. A su vez, usando síntesis de controladores mostramos como construir
automáticamente un controlador que garantiza que la actualización sucederá, y además, que dicha
actualización será de manera segura.\\
\end{tabular}

\subsection*{Carrera de Doctorado}

\begin{tabular}{L!{\VRule}R}
Título & \textbf{Adaptación vía Controladores Discretos Multi-Capa}\\
Director & Sebastián Uchitel (DC, FCEN-UBA)\\
Puntaje & Aprové materias por 16 puntos. Aun no pedidos.\\
& Alumno de doctorado de LaFHIS (Laboratorio de Fundamentos y Herramientas para 
la Ingeniería del Software).\\
\end{tabular}


\subsection*{Premios y Becas}

\begin{tabular}{L!{\VRule}R}
May 2016 & \textbf{Best Paper Award}, \textit{International Symposium on Software Engineering for Adaptive and
Self-Managing Systems (SEAMS'16)}\\

Abr 2015 Abr 2020 & \textbf{Beca doctoral}, \textit{Consejo Nacional de Investigaciones Científicas y Técnicas (CONICET)}, Argentina\\

%------------------------------------------------

Ago 2012 Ago 2014 & \textbf{Beca TIC}, Ministerio de Ciencia, Tecnología e Innovación Productiva, Argentina\\
& \vspace{-0.7cm} Obtuve una beca por dos años, otorgada por el MinCyT para obtener el título de 
Licenciado en Ciencias de la Computación. Los beneficiarios de este plan fueron
elegidos en base al número de exámenes rendidos, los años emprendidos en el estudio y el promedio de notas obtenidas.\\

\end{tabular}


\subsection*{Idiomas}

\begin{tabular}{L!{\VRule}R}
{\bf Español}&{\bf Nativo}\\
{\bf Inglés }&{\bf B2} (First Certificate in English 2006)\\
\end{tabular}

\end{document}


