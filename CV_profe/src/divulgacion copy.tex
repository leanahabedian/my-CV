
\item 2010 - Colaboraci\'on en la Semana de la Computaci\'on, FCEyN,
    UBA.

\item Dictado de charlas de divulgaci\'on en diferentes \'ambitos:

\begin{compactitem}[-]
\item[\bi{November}{Noviembre} 2010] ``Modelos, juegos y... S\'intesis de Controladores''\\
    Charlas de borrachos - seminario de difusi\'on a
    estudiantes de grado\\
    Departamento de Computaci\'on, FCEyN, Universidad de Buenos
    Aires.

\item[\bi{September}{Septiembre} 2010] ``Estrategias de control
    basadas en juegos infinitos entre
    dos jugadores''\\
    Escuela T�cnica Nro 28 Ditrito Escolar Nro 2 ``Rep\'ublica
    Francesa'',programa Exactas va a la Escuela.

\item[\bi{June}{Junio} 2010] ``Modelos, juegos y estrategias''\\
    Expotics - Primera feria exposici�n de ofertas educativas
    de nivel superior en tecnolog�as de la informaci�n y la
    comunicaci�n.\\
    Escuela T�cnica Nro 7 Ditrito Escolar Nro 5 ``Dolores
    Lavalle de Lavalle'', programa Exactas va a la Escuela.
\end{compactitem}


%\item Actuaci\'on como jurado de las siguientes actividades:
%
%\begin{compactitem}[-]
%\item Certamen Nacional de la Olimp\'{\i}ada Matem\'atica
%Argentina, 2004, 2006 y 2007.
%
%\item XVII Olimp\ACAAAA!!!!!!!!!!!!!!!!!!!!!!!
%
%'{\i}ada Matem\'atica
%    del
%    Cono
%    Sur,
%    Buenos
%    Aires,
%    mayo de 2006.
%
%\item Competencia de Programaci\'on de ACM interna de la FCEyN y
%Regional Sudamericana, en 2003, 2004 y 2005.
%
%\item XVIII Olimp\'{\i}ada Iberoamericana de Matem\'atica, Mar del
%Plata, septiembre de 2003.
%
%\item $\fuente 37^{ma}$ Feria Juvenil de Ciencia y Tecnolog\'{\i}a
%de la Ciudad de Buenos Aires, septiembre de 2003.
%
%\item XIII y XIV Olimp\'{\i}adas Intercolegiales ORT de
%Ma\-te\-m\'a\-ti\-ca, Buenos Aires, 1997 y 1998.
%\end{compactitem}
%
%\item Trabajo en la Fundaci\'on Olimp\'{\i}ada Matem\'atica
%Argentina, de abril de 1996 a septiembre de 1998. Funciones
%desempe\~nadas:
%
%\begin{compactitem}[-]
%\item Organizaci\'on y dictado de cursos de geometr\'{\i}a para
%profesores de en\-se\-\~nan\-za media y alumnos de alto
%rendimiento en olimp\'{\i}adas de matem\'atica, utilizando el
%software ``Cabri-G\'eom\`etre''. Cursos dictados en Buenos Aires,
%Rosario del Tala (Entre R\'{\i}os), Crespo (Entre R\'{\i}os),
%Bariloche (R\'{\i}o Negro) y Lincoln (Buenos Aires).
%
%\item Organizaci\'on y elaboraci\'on de problemas para
%competencias de resoluci\'on de problemas de geometr\'{\i}a
%utilizando el software ``Cabri-G\'eom\`etre''.
%
%\item Edici\'on de la publicaci\'on mensual ``Lugar
%Geom\'etrico''.
%\end{compactitem}
