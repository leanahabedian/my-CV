%%%%%%%%% DC

%Varios temas del \'area de Ingenier\'ia de Software forman parte de
%la curr\'icula de Cs. de la Computaci\'on de la ACM, as\'i como de
%los programas de las carreras de Cs. de la Computaci\'on en
%universidades de todo el mundo.

%Dentro de las materias de la Licenciatura en Ciencias de
%la Computaci\'on de la FCEyN, UBA, �stos temas se abordan en materias como 
%Bases de Datos, Ingenier\'ia de Software 1 y 2,
%Administraci\'on de proyectos inform�ticos y An\'lisis de Requerimientos
%Temporales entre otras.


Teniendo en cuenta mi historial docente y siendo egresado del actual plan de 
estudios, como Profesor Adjunto en el Departamento de Computaci�n estoy en 
condiciones de dictar cualquier materia que el Departamento considere 
necesario. 

Sin embargo, dado que en mis �ltimos a�os como docente de la FCEyN me he desarrollado principalmente en el �rea de Ingenier�a de Software y que a esta �ltima tambi�n corresponde mi �rea de investigaci�n voy a realizar una propuesta para mejorar la oferta acad�mica del �rea. 

Si bien Ingenier\'ia de Software I incluye hoy en su programa una extensa introducci\'on a la Ingenier\'ia de Software, considero que en la materia hay lugar para presentar tanto los problemas cl\'asicos del area, como tambi\'en un panorama sobre distintas t�cnicas y herramientas 
que hacen al estado del arte de la Ingenier\'ia de Software. 
Con esto en mente, concetrar\'ia mis esfuerzos en no solo mantener el gran nivel que la materia tiene hoy, sino en avanzar sobre algunas de estas  t�cnicas y herramientas tienen especial potencial para la interacci\'on cient\'ifico tecnol\'ogica. 
Para esto, me propongo profundizar en la materia el uso de t\'ecnicas de an\'alisis, relevamiento y
validaci\'on de requerimientos provenientes de la ingenier\'ia de requerimientos, y t\'ecnicas de
validaci\'on y verficaci\'on basadas en modelos de comportamiento como por ejemplo Model Checking. Adem\'as presentar\'ia un panorama sobre t\'ecnicas de generaci\'on autom�tica de casos de prueba en el marco t\'ecnicas de testing automatizado. 

Cabe destacar que los cambios que propongo en la materia est�n motivados no solo en mi experiencia como investgador en el �rea de Ingenier�a de Software, sino tambi�n en mi conocimiento de la materia dado por mi trabajo como docente en la misma. 
Los alumnos de la materia son estudiantes jovenes y en su mayoria trabajadores en la industria del software. 
En este contexto, motivarlos a los alumnos a la aplicaci�n de dichas t�cnicas y herramientas 
es clave para aumentar la adopci�n en la industria de propuestas con tanto potencial como la 
generaci�n autom�tica de casos de prueba. Todo esto con miras a mejorar la calidad del software que se produce en el pais y, porque no, incrementalmente mejorar la posici�n del pais en cuanto a la industra del software a nivel mundial. 
Esto �ltimo es de gran importancia en un contexto como el argentino en el cual muchos sectores declarados de relavancia nacional, como por ejemplo AgroTICs, Energ�a, y Transporte, podr�an 
beneficiarse con la adopci�n de nuevas t�cnicas de elaboraci�n de requeriminetos, verificaci�n de software, y testing automatizado. 

Por �ltimo, en cuanto a materias optativas, considero muy importante la ampliaci�n y actualizaci�n de la oferta actual, especialmente en el �rea docente de Teor�a. 
En particular, desde mi experiencia planeo proponer una materia optativa basada en mis temas de investigaci�n, ya sea en t�cnicas de control autom�tico o bien en la aplicaci�n de dichas t�cnicas a la rob�tica, �rea en la cual estoy trabajando con un grupo de alumnos del departamento. 



%%%%%%%%%%%%%%%
%Para las materias optativas propongo utilizar principalmente la
%modalidad de seminario, alternando clases te�ricas con exposici�n
%de temas por parte de los alumnos. El objetivo es profundizar en
%ciertos problemas de inter�s actual, que variar\'{\i}an en cada
%cuatrimestre en que se dicte la materia, y poniendo especial
%\'enfasis en las preguntas abiertas alrededor de dichos problemas.
%Se espera que el seminario despierte el inter�s de los alumnos,
%dando lugar a nuevas Tesis de Licenciatura y Doctorado en el �rea
%y la incorporaci\'on de los mismos al grupo de investigaci\'on.
%
%
%%completar
%
