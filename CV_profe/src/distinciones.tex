
\item ``Premio al M\'erito'', otorgado por el Concejo Municipal de
San Carlos de Ba\-ri\-lo\-che en agosto de 2003 (declaraci\'on
917-CM-03).

\item { ACM Programming Contest}

\begin{compactitem}

\item World Finals 2003, Hollywood, USA: integrante del equipo de
la UBA que obtuvo el $\fuente 12^{do}$ lugar (medalla de Bronce)
resultando adem\'as cam\-pe\'on Americano.

\item World Finals 2002, Honolulu, Hawaii: integrante del equipo
de la UBA que obtuvo el $\fuente 10^{mo}$ lugar (medalla de
Bronce) resultando adem\'as cam\-pe\'on Latinoamericano.

\item Regional Sudamericano 2001: $\fuente 2^{do}$ lugar.

\item Regional Sudamericano 2000: $\fuente 3^{er}$ lugar.

\end{compactitem}

\item { Competencia Ernesto Paenza de Matem\'atica}
\begin{compactitem}
\item 2002: $\fuente 8^{vo}$ lugar, menci\'on de Honor. \item
2001: $\fuente 6^{to}$ lugar, menci\'on de Honor.
\end{compactitem}

\item { Asian Pacific Mathematics Olympiad}
\begin{compactitem}
\item 1996: medalla de Bronce. \item 1995: menci\'on de Honor.
\end{compactitem}

\item { Olimp\'{\i}ada Iberoamericana de Matem\'atica}
\begin{compactitem}
\item 1995, Santiago, Chile: medalla de Oro.
\end{compactitem}

\item { International Mathematical Olympiad}
\begin{compactitem}
\item 1995, Toronto, Canad\'a: miembro del equipo Argentino.
\end{compactitem}

\item { Torneo Internacional de las Ciudades}
\begin{compactitem}
\item 1994 y 1995: diploma de Honor.
\end{compactitem}

\item { Olimp\'{\i}ada Rioplatense de Matem\'atica}
\begin{compactitem}
\item 1992, Maldonado, Uruguay: $\fuente 2^{do}$ lugar, medalla de
Plata.
\end{compactitem}

\item { Olimp\'{\i}ada Matem\'atica Argentina}
\begin{compactitem}
\item 1993, 1994 y 1995: menci\'on de Honor. \item 1992: $\fuente
2^{do}$ lugar, primer subcampeona Nacional.
\end{compactitem}
